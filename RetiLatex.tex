\documentclass[11pt, oneside]{article}   	% use "amsart" instead of "article" for AMSLaTeX format
\usepackage{geometry}                		% See geometry.pdf to learn the layout options. There are lots.
\geometry{letterpaper}                   		% ... or a4paper or a5paper or ... 
%\geometry{landscape}                		% Activate for rotated page geometry
\usepackage[parfill]{parskip}    			% Activate to begin paragraphs with an empty line rather than an indent
\usepackage{graphicx}				% Use pdf, png, jpg, or eps§ with pdflatex; use eps in DVI mode
								% TeX will automatically convert eps --> pdf in pdflatex		
\usepackage{amssymb}
\usepackage{mathtools}
\usepackage{enumerate}
\usepackage{tikz}

\usetikzlibrary{arrows}

\def\firstcircle{(90:1.75cm) circle (2.5cm)}
\def\secondcircle{(210:1.75cm) circle (2.5cm)}
\def\thirdcircle{(330:1.75cm) circle (2.5cm)}

%SetFonts

%SetFonts


\title{Domande Reti degli Elaboratori}
\author{Colacel Alexandru Andrei\\ \texttt{colacel.1941345@studenti.uniroma1.it}}
\date{AA 2021/2022}							% Activate to display a given date or no date

\begin{document}

\maketitle

\textbf{Quali livelli nella pila dei protocolli di internet vengono elaborati da un router? Quali da un commutatore a livello di collegamento? Quali da un host?}\\


\textbf{Che cos'è un messaggio a livello di applicazione? Un segmento a livello di trasporto? Un datagramma a livello di rete? Un frame a livello di collegamento?}\\ 	
	
\textbf{Alcuni fornitori di sevizi hanno creato reti private. Si descriva la rete di Google. Quali sono le motivazioni che li spingono a crearsi le proprie reti?}\\

\textbf{Descivi le differenze tra il protocollo TCP e UDP}\\

\textbf{Descrivi il protocollo UDP}\\

\textbf{Descrivi il protocollo TCP}\\

\textbf{Nel protocollo di trasporto TCP cosa si intende per temporizzazione?}\\

\textbf{Nel protocollo di trasporto TCP cosa si intende per throughput?}\\

\textbf{Nel protocollo di trasporto TCP cosa si intende per Affidabilità?}\\

\textbf{Nel protocollo di trasporto TCP cosa si intende per sicurezza?}\\

\textbf{Quali sono i quattro punti fondamentali del protocollo di trasporto TCP?}\\

\textbf{Descrivi le caratteristiche dei processi comunicanti e il ruolo del progettista}\\

\textbf{Cosa si intende per architettura client-server e cosa per architettura P2P?}\\

\textbf{Considerare l’invio di un pacchetto da un host a un altro lungo un percorso fisso ed elencare tutte le componenti di ritardo nel ritardo complessivo. Quali sono le costanti e quali sono le variabili?}\\

\textbf{Si descriva il funzionamento dell’applicazione Web in quale livello e per cosa è usata? Descrivi il protocollo usato per scambiare i mesasggi. Si spieghi cosa sono i cookies. Discutere anche come si possono migliorare le prestazioni dell’applicazione web}\\

\textbf{Quali sono le prestazioni in termini di latenza sperimentale su tale linnk nel caso di un processo di arrivo del traffico Poissoniano al crescere del carico offerto dal link?}\\

\textbf{Quali sono i vantaggi di avere uno stack protocollare basato sulla stratificazione? (architettura a livelli)}\\

\textbf{Quali sono i livelli protocollari presenti su un host?}\\

\textbf{Funzionamento e le problematiche legate al formato dei messaggi e gli elementi di rete coinvolti}\\

\textbf{Discutere l'impatto che la lunghezza di un pacchetto ha sulle prestazioni del sistema, discutendo l''effetto pipeling, il caso di un canale rumoroso ecc}\\

\textbf{Si discutano le differenza tra una rete a circuito e a pacchetto}\\

\textbf{Cosa si intende per modello OSI e cosa per ISO ?}\\

\textbf{Descrivi il livello fisico}\\

\textbf{Descrivi il livello di collegamento}\\

\textbf{Descrivi il livello di rete}\\

\textbf{Descrivi il livello di trasporto}\\

\textbf{Descrivi il livello di applicazione}\\

\textbf{Quali sono ii 5 livelli della pila dei protocolli?}\\

\textbf{Cos’è la stratificazione dei protocolli? Vantaggi + Svantaggi + i 5 livelli}\\

\textbf{Cos'è la modularità nell'architettura a livelli?}\\

\textbf{Cosa si intende per architettura a livelli?}\\

\textbf{Cos'è il thorughput? Descrivi le due classificazioni e i problemi che possono verificarsi}\\

\textbf{Descrivi il ritardo di pacchettizzazione}\\

\textbf{Descrivi il ritardo end-to-end}\\

\textbf{Descrivi il ritardo di propagazione}\\

\textbf{Descrivi il ritardo di trasmissione}\\

\textbf{Descrivi il ritardo di accodamento}\\

\textbf{Descrivi il ritardo di Elaborazione}\\

\textbf{Descrivi in generale i ritardi in una connessione}\\

\textbf{Descrivi la rete di reti con le strutture 1, 2, 3, 4, 5}\\

\textbf{Descrivi i due tipi di multiplexing nelle reti a commutazione di circuito}\\

\textbf{Descrivi la commutazione di circuito e le varie caratteristiche}\\

\textbf{Quali sono le differenze. tra commutazione di circuito e commutazione di pacchetto?}\\

\textbf{Quali tecniche si usano per spostare i dati nella rete?}\\

\textbf{Descrivi le tabelle di inoltro e il loro funzionamento}\\

\textbf{Cosa sono i ritardi di accodamento?}\\

\textbf{Cosa sono i ritardi di accodamento?}\\

\textbf{Quali tipo di trasmissione usano i commutatori di pacchetto e come funziona esattamente}\\

\textbf{Descrivi i canali radio satellitari}\\

\textbf{Descrivi i canali radio terrestri}\\

\textbf{Descrivi la fibra ottica}\\

\textbf{Descrivi il cavo coassiale}\\

\textbf{Descrivi il doppino di rame intrecciato}\\

\textbf{Descrivi la suddivisione dei mezzi trasmissivi e quali modi esistono per trasmettere un bit}\\

\textbf{Quali sono e come funzionano i mezzi trasmissivi?}\\

\textbf{Differnz tra accesso alla rete locale e accesso wireless su scala geografica}\\

\textbf{Descrivi l'utilizzo di internet tramite satellite}\\

\textbf{Descrivi l’utilizzo del cable modem}\\

\textbf{Definisci l’acronimo di FTTH e definisci il suo utilizzo}\\

\textbf{Descrivi cosa sono le reti di accesso e dove vengono utilizzate maggiormente?}\\

\textbf{Dai la definizione di protocollo di internet}\\

\textbf{Descrivi i due standard di internet}\\

\textbf{Descrivi i due principali protocolli di internet}\\

\textbf{Descrivi l'ISP, provider, API e i vari usi}\\

\textbf{Cos'è il commutatore di pacchetto?}\\

\textbf{Come sono connessi i dispositivi di internet?}\\

\textbf{Come sono detti i dispositivi di internet?}\\

\textbf{Vengono configurate manualmente in ciascun router o Internet impiega una procedura automatizzata?}\\

\textbf{Come vengono impostate le tabelle di inoltro?}\\

\textbf{Ma come fa il router a determinare su quale collegamento il pacchetto deve essere inoltrato?}\\

\textbf{Che cos'è un protocollo?}\\

\textbf{Come può connettersi un tosta pane o un sensore meteorologico?}\\

\textbf{Che cos'è un'applicazione distribuita?}\\

\textbf{Quali tipi di collegamenti sono presenti in internet?}\\

\textbf{Che cosa sono i router?}\\

\textbf{Che cosa sono i commutatori di pacchetto e TCP/IP?}\\




\end{document}  